% A few common packages
\usepackage{amsmath}
\usepackage{amsthm}
\usepackage{array}
\usepackage{graphicx}
\usepackage{relsize}
\usepackage[dvipsnames]{xcolor}
\usepackage[titletoc]{appendix}

% Some other useful packages
\usepackage{caption}
\usepackage{subcaption}  % \begin{subfigure}...\end{subfigure} within figure
\usepackage{multirow}
\usepackage{tabularx}

% Uncomment the following to attempt to enforce Type 1 or TrueType 
% fonts. ProQuest does not want the type 3 fonts used by default as
% of Dec. 2019 - see 
% https://support.proquest.com/articledetail?id=kA01W000000k9o2SAA . 
% If you are unable to embed fonts such as 'Zapf Dingbats' or 
% 'Symbol', try using raster images (.jpg or .png) instead of vector 
%images (.pdf or .eps).
% \usepackage[T1]{fontenc} 

% plainpages=false fixes the "duplicate ignored" error with page counters
% Set pdfborder to 0 0 0 to disable colored borders around PDF hyperlinks
\usepackage[plainpages=false,pdfborder={0 0 0}]{hyperref}

% Uncomment the following line to use the algorithm package,
% which provides an algorithm environment similar to figure and table
% ("\begin{algorithm}...\end{algorithm}"). A list of algorithms will
% automatically be added in the preliminary pages. Note that you
% probably want a package for the actual code to go with this (e.g.,
% algorithmic).
%\usepackage{algorithm}

% Uncomment the following line to enable Unicode support. This will allow you
% to enter non-ASCII characters (such as accented characters) directly without
% having to use LaTeX's awkward escape syntax (e.g., \'{e})
% NOTE: You may have to install the ucs.sty package for this to work. See:
% http://www.unruh.de/DniQ/latex/unicode/
%\usepackage[utf8x]{inputenc}

\usepackage{apacite}

% Uncomment the following to avoid "widowing", where page breaks cause
% single lines of paragraphs to float onto the next page (this is not
% a UCI requirement but more of an aesthetic choice).
%\widowpenalty=10000
%\clubpenalty=10000

% Modify or extend these at will.
\newtheorem{theorem}{\textsc{Theorem}}[chapter]
\newtheorem{definition}{\textsc{Definition}}[chapter]
\newtheorem{example}{\textsc{Example}}[chapter]

\newcommand{\field}[1]{{\color{BrickRed}\textbf{#1}}}

\newcommand{\dissertationsection}[1]{\chapter{#1}}